% Logical markup

% These commands should be used to help make the source easy to understand
% and consistently typeset.
% Search for them in the source files to see how to use them.

% Tone of voice
\newcommand{\shout}[1]{\textsc{#1}}
\newcommand{\scream}[1]{\MakeUppercase{#1}}
\newcommand{\prophesy}[1]{\shout{#1}}

% parsel
\newcommand{\parselify}[1]{%
  \StrSubstitute{#1}{ss}{ß}[\parselstring]%
  \StrSubstitute{\parselstring}{s}{ss}[\parsselstring]%
  \StrSubstitute{\parsselstring}{ß}{sss}[\parssselstring]%
}
% N.B. Other commands, such as \emph, cannot be used inside \parsel
\newcommand{\parsel}[1]{\parselify{\fontspec[ExternalLocation]{Parseltongue.ttf}#1}{\ptsansi\parssselstring}}

% Special types of text
\newcommand{\abbrev}[1]{\textsc{\MakeLowercase{#1}}\xspace}

% Common abbreviations
\newcommand{\am}{~a.m.\xspace}
\renewcommand{\pm}{~p.m.\xspace}
\newcommand{\SPHEW}{\abbrev{SPHEW}}

% Newspaper headlines
\newcommand{\headline}[1]{\begin{center}\textsc{#1}\end{center}}
\newcommand{\inlineheadline}[1]{\textsc{#1}}
\newenvironment{headlines}{
  \newcommand{\header}[1]{\begin{SingleSpace}\upshape ##1\end{SingleSpace}}
  \let\hmorSavedLabel\label
  \renewcommand{\label}[1]{{\upshape\itshape ##1}}%
  \begin{Spacing}{0.75}
  \begin{center}
  \scshape
  \nonzeroparskip
}{
  \end{center}
  \end{Spacing}
  \let\label\hmorSavedLabel
}

% Author’s notes
\newcommand{\authorsnotefootnotemark}{\footnotemark}
% \newcommand{\authorsnotetext}[1]{\footnotetext{Author’s note: #1}}
\newcommand{\authorsnotetext}[1]{}

% Letters
\newcommand{\letterAddress}[1]{\pagebreak[1]\noindent{}\fontspec[ExternalLocation]{Graphe_Alpha_alt.ttf}\scriptsize#1\nopagebreak[4]\par}
\newcommand{\letterClosing}[2][\vskip 1\baselineskip]{\nopagebreak[4]\fontspec[ExternalLocation]{Graphe_Alpha_alt.ttf}\scriptsize#1\par\nopagebreak[5]\noindent#2}

\newenvironment{writtenNote}{\fontspec[ExternalLocation]{Graphe_Alpha_alt.ttf}\scriptsize%
  \renewcommand{\emph}{\uline}
  \vskip 1\baselineskip plus 1\baselineskip minus 1\baselineskip%
  \begin{adjustwidth}{\parindent}{\parindent}%
    \par\noindent\itshape}
  {\end{adjustwidth}\vskip 1\baselineskip plus 1\baselineskip minus 1\baselineskip}

\newenvironment{chapterOpeningQuote}{%
  \parindent=0pt%
  \itshape}
  {%
  \newline\rule[1ex]{\textwidth}{.1pt}\newline%
  }

% Definition of chapterOpeningAuthorNote when they are desired to be printed
% FIXME: Make the environment definition switchable with a flag
% \newenvironment{chapterOpeningAuthorNote}{%
% \parindent=0pt%
% \itshape
% E.~Y.:~
% }
% {%
% \newline\rule[1ex]{\textwidth}{.1pt}\newline%
% }
\excludecomment{chapterOpeningAuthorNote}


% Chapter parts
% \partchapter{The Stanford Prison Experiment}{TSPE}{XIII}{Aftermaths}
% TOC: TSPE part XIII: Aftermaths
% Page header: The Stanford Prison Experiment XIII: \\? Aftermaths
% Title: The Stanford Prison Experiment, Part XIII: \\? Aftermaths
\newcommand{\namedpartchapter}[5][\relax]{%
	\chapter[%
			\texorpdfstring{%
				\abbrev{#3, part #4}: #5}{%
				#3, part #4: #5}][%
			\mbox{#2 #4:} \mbox{#5}]{%
			#2, Part~#4:\protect\linebreak[1] #5#1}%
}

\newcommand{\partchapter}[3][\relax]{%
	\chapter[\texorpdfstring{#2, \abbrev{part #3}}{#2, part #3}]%
		[#2 #3]{#2, Part~#3#1}}

% Hanging paras for play scripts (used in Omake IV)
\newenvironment{playdialog}{\begin{hangparas}{2em}{1}}{\end{hangparas}}


% Stars

% Single “magic star” decoration
\newcommand{\Star}{{\fontspec[ExternalLocation]{Miscelanea.ttf}*}}

% Three “magic stars” decoration
\def\Stars{{\large\Star\kern-.6ex\lower1.3ex\hbox{\large\Star}\kern-.1ex\raise.2ex\hbox{\tiny\Star}\spacefactor1000}}

% Text break made of \Stars (only used to define other commands)
\makeatletter
\def\sbre@k{\mbox{}\nobreak\hfill\mbox{}\allowbreak\rule{.60\textwidth}{.0pt}\par%
  \vskip 0pt plus 2\baselineskip\noindent{%
    \parbox[c][0pt][c]{\textwidth}{%
      \hfil \hbox{\lower14pt\hbox{\normalsize\Stars}}%
    }%
  }}

% A standalone break
\def\later{\sbre@k%
  \vskip 0pt plus 2\baselineskip%
  \par\rule{.5\textwidth}{.0pt}\vskip1pt\noindent}

% A break followed by a new section
\newcommand{\latersection}[1]{\sbre@k\section{#1}}
\makeatother
