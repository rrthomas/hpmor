\chapter{Comparing Reality To Its Alternatives}

\begin{chapterOpeningAuthorNote}
If J.~K.~Rowling asks you about this story, you know nothing.
\end{chapterOpeningAuthorNote}
\begin{chapterOpeningQuote}
But then the question is—who?
\end{chapterOpeningQuote}

\lettrine[ante=“]{G}{ood} Lord,” said the barman, peering at Harry, “is this—can this be—?”

Harry leaned towards the bar of the Leaky Cauldron as best he could, though it came up to somewhere around the tips of his eyebrows. A question like \emph{that} deserved his very best.

“Am I—could I be—maybe—you never know—if I’m \emph{not}—but then the question is—\emph{who?}”

“Bless my soul,” whispered the old barman. “Harry Potter…what an honour.”

Harry blinked, then rallied. “Well, yes, you’re quite perceptive; most people don’t realise that so quickly—”

“That’s enough,” Professor McGonagall said. Her hand tightened on Harry’s shoulder. “Don’t pester the boy, Tom, he’s new to all this.”

“But it is him?” quavered an old woman. “It’s Harry Potter?” With a scraping sound, she got up from her chair.

“Doris—” McGonagall said warningly. The glare she shot around the room should have been enough to intimidate anyone.

“I only want to shake his hand,” the woman whispered. She bent low and stuck out a wrinkled hand, which Harry, feeling confused and more uncomfortable than he ever had in his life, carefully shook. Tears fell from the woman’s eyes onto their clasped hands. “My grandson was an Auror,” she whispered to him. “Died in seventy-nine. Thank you, Harry Potter. Thank heavens for you.”

“You’re welcome,” Harry said automatically, and then he turned his head and shot Professor McGonagall a frightened, pleading look.

Professor McGonagall slammed her foot down just as the general rush was about to start. It made a noise that gave Harry a new referent for the phrase “Crack of Doom”, and everyone froze in place.

“We’re in a hurry,” Professor McGonagall said in a voice that sounded perfectly, utterly normal.

They left the bar without any trouble.

“Professor?” Harry said, once they were in the courtyard. He had meant to ask what was going on, but oddly found himself asking an entirely different question instead. “Who was that pale man, by the corner? The man with the twitching eye?”

“Hm?” said Professor McGonagall, sounding a bit surprised; perhaps she hadn’t expected that question either. “That was Professor Quirinus Quirrell. He’ll be teaching Defence Against the Dark Arts this year at Hogwarts.”

“I had the strangest feeling that I knew him…” Harry rubbed his forehead. “And that I shouldn’t shake his hand.” Like meeting someone who had been a friend, once, before something went drastically wrong…that wasn’t really it at all, but Harry couldn’t find words. “And what \emph{was}…all of that?”

Professor McGonagall was giving him an odd glance. “Mr~Potter…do you know…how \emph{much} have you been told…about how your parents died?”

Harry returned a steady look. “My parents are alive and well, and they always refused to talk about how my \emph{genetic} parents died. From which I infer that it wasn’t good.”

“An admirable loyalty,” said Professor McGonagall. Her voice went low. “Though it hurts a little to hear you say it like that. Lily and James were friends of mine.”

Harry looked away, suddenly ashamed. “I’m sorry,” he said in a small voice. “But I \emph{have} a Mum and Dad. And I know that I’d just make myself unhappy by comparing that reality to…something perfect that I built up in my imagination.”

“That is amazingly wise of you,” Professor McGonagall said quietly. “But your \emph{genetic} parents died very well indeed, protecting you.”

\emph{Protecting me?}

Something strange clutched at Harry’s heart. “What…\emph{did} happen?”

Professor McGonagall sighed. Her wand tapped Harry’s forehead, and his vision blurred for a moment. “Something of a disguise,” she said, “so that this doesn’t happen again, not until you’re ready.” Then her wand licked out again, and tapped three times on a brick wall…

…which hollowed into a hole, and dilated and expanded and shivered into a huge archway, revealing a long row of shops with signs advertising cauldrons and dragon livers.

Harry didn’t blink. It wasn’t like anyone was turning into a cat.

And they walked forwards, together, into the wizarding world.

There were merchants hawking Bounce Boots (“Made with real Flubber!”) and “Knives +3! Forks +2! Spoons with a +4 bonus!” There were goggles that would turn anything you looked at green, and a line-up of comfy armchairs with ejection seats for emergencies.

Harry’s head kept rotating, rotating like it was trying to wind itself off his neck. It was like walking through the magical items section of an \emph{Advanced Dungeons and Dragons} rulebook (he didn’t play the game, but he did enjoy reading the rulebooks). Harry desperately didn’t want to miss a single item for sale, in case it was one of the three you needed to complete the cycle of infinite \emph{wish} spells.

Then Harry spotted something that made him, entirely without thinking, veer off from the Deputy Headmistress and start heading straight into the shop, a front of blue bricks with bronze-metal trim. He was brought back to reality only when Professor McGonagall stepped right in front of him.

“Mr~Potter?” she said.

Harry blinked, then realised what he’d just done. “I’m sorry! I forgot for a moment that I was with you instead of my family.” Harry gestured at the shop window, which displayed fiery letters that shone piercingly bright and yet remote, spelling out \emph{Bigbam’s Brilliant Books}. “When you walk past a bookshop you haven’t visited before, you have to go in and look around. That’s the family rule.”

“That is the most Ravenclaw thing I have ever heard.”

“What?”

“Nothing. Mr~Potter, our first step is to visit Gringotts, the bank of the wizarding world. Your \emph{genetic} family vault is there, with the inheritance your \emph{genetic} parents left you, and you’ll need money for school supplies.” She sighed. “And, I suppose, a certain amount of spending money for books could be excused as well. Though you might want to hold off for a time. Hogwarts has quite a large library on magical subjects. And the tower in which, I strongly suspect, you will be living, has a more broad-ranging library of its own. Any book you bought now would probably be a duplicate.”

Harry nodded, and they walked on.

“Don’t get me wrong, it’s a \emph{great} distraction,” Harry said as his head kept swivelling, “probably the best distraction anyone has ever tried on me, but don’t think I’ve forgotten about our pending discussion.”

Professor McGonagall sighed. “Your parents—or your mother at any rate—may have been very wise not to tell you.”

“So you wish that I could continue in blissful ignorance? There is a certain flaw in that plan, Professor McGonagall.”

“I suppose it would be rather pointless,” the witch said tightly, “when anyone on the street could tell you the story. Very well.”

And she told him of He-Who-Must-Not-Be-Named, the Dark Lord, Voldemort.

“Voldemort?” Harry whispered. It should have been funny, but it wasn’t. The name burned with a cold feeling, ruthlessness, diamond clarity, a hammer of pure titanium descending upon an anvil of yielding flesh. A chill swept over Harry even as he pronounced the word, and he resolved then and there to use safer terms like You-Know-Who.

The Dark Lord had raged upon wizarding Britain like a wilding wolf, tearing and rending at the fabric of their everyday lives. Other countries had wrung their hands but hesitated to intervene, whether out of apathetic selfishness or simple fear, for whichever was first among them to oppose the Dark Lord, their peace would be the next target of his terror.

(\emph{The bystander effect,} thought Harry, thinking of Latané and Darley’s experiment which had shown that you were more likely to get help if you had an epileptic fit in front of one person than in front of three. \emph{Diffusion of responsibility, everyone hoping that someone else would go first.})

The Death Eaters had followed in the Dark Lord’s wake and in his vanguard, carrion vultures to pick at wounds, or snakes to bite and weaken. The Death Eaters were not as terrible as the Dark Lord, but they were terrible, and they were many. And the Death Eaters wielded more than wands; there was wealth within those masked ranks, and political power, and secrets held in blackmail, to paralyse a society trying to protect itself.

An old and respected journalist, Yermy Wibble, called for increased taxes and conscription. He shouted that it was absurd for the many to cower in fear of the few. His skin, only his skin, had been found nailed to the newsroom wall that next morning, next to the skins of his wife and two daughters. Everyone wished for something more to be done, and no-one dared take the lead to propose it. Whoever stood out the most became the next example.

Until the names of James and Lily Potter rose to the top of that list.

And those two might have died with their wands in their hands and not regretted their choices, for they \emph{were} heroes; but for that they had an infant child, their son, Harry Potter.

Tears were coming into Harry’s eyes. He wiped them away in anger or maybe desperation, \emph{I didn’t know those people, not really, they aren’t my parents \emph{now}, it would be pointless to feel so sad for them—}

When Harry was done sobbing into the witch’s robes, he looked up, and felt a little bit better to see tears in Professor McGonagall’s eyes as well.

“So what happened?” Harry said, his voice trembling.

“The Dark Lord came to Godric’s Hollow,” Professor McGonagall said in a whisper. “You should have been hidden, but you were betrayed. The Dark Lord killed James, and he killed Lily, and he came in the end to you, to your cot. He cast the Killing Curse at you, and that was where it ended. The Killing Curse is formed of pure hate, and strikes directly at the soul, severing it from the body. It cannot be blocked, and whomever it strikes, they die. But you survived. You are the only person ever to survive. The Killing Curse rebounded and struck the Dark Lord, leaving only the burnt hulk of his body and a scar upon your forehead. That was the end of the terror, and we were free. That, Harry Potter, is why people want to see the scar on your forehead, and why they want to shake your hand.”

The storm of weeping that had washed through Harry had used up all his tears; he could not cry again, he was done.

(And somewhere in the back of his mind was a small, small note of confusion, a sense of something wrong about that story; and it should have been a part of Harry’s art to notice that tiny note, but he was distracted. For it is a sad rule that whenever you are most in need of your art as a rationalist, that is when you are most likely to forget it.)

Harry detached himself from Professor McGonagall’s side. “I’ll—have to think about this,” he said, trying to keep his voice under control. He stared at his shoes. “Um. You can go ahead and call them my parents, if you want, you don’t have to say ‘genetic parents’ or anything. I guess there’s no reason I can’t have two mothers and two fathers.”

There was no sound from Professor McGonagall.

And they walked together in silence, until they came before a great white building with vast bronze doors, and carven words above saying \emph{Gringotts Bank.}

%  LocalWords:  ood Bigbam’s
