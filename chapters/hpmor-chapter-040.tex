\partchapter{Pretending to be Wise}{II}

\lettrine{H}{arry}, holding the tea cup in the exactly correct way that Professor Quirrell had needed to demonstrate three times, took a small, careful sip. All the way across the long, wide table that was the centrepiece of Mary’s Room, Professor Quirrell took a sip from his own cup, making it look far more natural and elegant. The tea itself was something whose name Harry couldn’t even pronounce, or at least, every time Harry had tried to repeat the Chinese words, Professor Quirrell had corrected him, until finally Harry had given up.

Harry had manœuvred himself into getting a glimpse at the bill last time, and Professor Quirrell had let him get away with it.

He’d felt an impulse to drink a Comed-Tea first.

\emph{Even taking that into account}, Harry had still been shocked out of his skin.

And it still tasted to him like, well, tea.

There was a quiet, nagging suspicion in Harry’s mind that Professor Quirrell \emph{knew} this, and was deliberately buying ridiculously expensive tea that Harry couldn’t appreciate \emph{just to mess with him.} Professor Quirrell \emph{himself} might not like it all that much. Maybe \emph{nobody} actually liked this tea, and the only point of it was to be ridiculously expensive and make the victim feel unappreciative. In fact, maybe it was really just ordinary tea, only you asked for it in a certain code, and they put a fake gigantic price on the bill…

Professor Quirrell’s expression was drawn and thoughtful. “No,” Professor Quirrell said, “you should \emph{not} have told the Headmaster about your conversation with Lord Malfoy. Please try to think faster next time, Mr~Potter.”

“I’m sorry, Professor Quirrell,” Harry said meekly. “I still don’t see it.” There were times when Harry felt very much like an impostor, pretending to be cunning in Professor Quirrell’s presence.

“Lord Malfoy is Albus Dumbledore’s opponent,” said Professor Quirrell. “At least for this present time. All Britain is their chessboard, all wizards their pieces. Consider: Lord Malfoy threatened to throw away everything, abandon his game, to take vengeance on you if Mr~Malfoy was hurt. In which case, Mr~Potter…?”

It took more long seconds for Harry to get it, but it was clear that Professor Quirrell wasn’t going to give any more hints, not that Harry wanted them.

Then Harry’s mind finally made the connection, and he frowned. “Dumbledore kills Draco, makes it look like \emph{I} did it, and Lucius sacrifices his game against Dumbledore to get at me? That…doesn’t seem like the Headmaster’s \emph{style}, Professor Quirrell…” Harry’s mind flashed back to a similar warning from Draco, which had made Harry say the same thing.

Professor Quirrell shrugged, and sipped his tea.

Harry sipped his own tea, and sat in silence. The tablecloth spread over the table was in a very peaceful pattern, seeming at first like plain cloth, but if you stared at it long enough, or kept silent long enough, you started to see a faint tracery of flowers glimmering on it; the curtains of the room had changed their pattern to match, and seemed to shimmer as though in a silent breeze. Professor Quirrell was in a contemplative mood that Saturday, and so was Harry, and Mary’s Room, it seemed, had not neglected to notice this.

“Professor Quirrell,” Harry said suddenly, “is there an afterlife?”

Harry had chosen the question carefully. Not, \emph{do you believe in an afterlife?} but simply \emph{Is there an afterlife?} What people \emph{really} believed didn’t seem to them like \emph{beliefs} at all. People didn’t say, ‘I strongly believe in the sky being blue!’ They just said, ‘the sky is blue’. Your true inner map of the world just felt to you like the way the world \emph{was…}

The Defence Professor raised his cup to his lips again before answering. His face was thoughtful. “If there is, Mr~Potter,” said Professor Quirrell, “then quite a few wizards have wasted a great deal of effort in their searches for immortality.”

“That’s not actually an answer,” Harry observed. He’d learned by now to notice that sort of thing when talking to Professor Quirrell.

Professor Quirrell set down his teacup with a small, high-pitched tacking sound on his saucer. “Some of those wizards were reasonably intelligent, Mr~Potter, so you may take it that the existence of an afterlife is not obvious. I have looked into the matter myself. There have been many claims of the sort which hope and fear would be expected to produce. Among those reports whose veracity is not in doubt, there is nothing which could not be the result of mere wizardry. There are certain devices said to communicate with the dead, but these, I suspect, only project an image from the mind; the result seems indistinguishable from memory because it \emph{is} memory. The alleged spirits tell no secrets they knew in life, nor could have learned after death, which are not known to the wielder—”

“Which is why the Resurrection Stone is not the most valuable magical artefact in the world,” said Harry.

“Precisely,” said Professor Quirrell, “though I wouldn’t say no to a chance to try it.” There was a dry, thin smile on his lips; and something colder, more distant, in his eyes. “You spoke to Dumbledore of that as well, I take it.”

Harry nodded.

The curtains were taking on a faintly blue pattern, and a dim tracery of elaborate snowflakes now seemed to be becoming visible on the tablecloth. Professor Quirrell’s voice sounded very calm. “The Headmaster can be very persuasive, Mr~Potter. I hope he has not persuaded you.”

“\emph{Heck} no,” said Harry. “Didn’t fool me for a second.”

“I should hope not,” said Professor Quirrell, still in that very calm tone. “I would be extremely put out to discover that the Headmaster had convinced you to throw away your life on some fool plot by telling you that death is the next great adventure.”

“I don’t think the Headmaster believed it himself, actually,” Harry said. He sipped his own tea again. “He asked me what I could possibly do with eternity, gave me the usual line about it being boring, and he didn’t seem to see any conflict between that and his own claim to have an immortal soul. In fact, he gave me a whole long lecture about how awful it was to want immortality before he claimed to have an immortal soul. I can’t quite visualize what must have been going on inside his head, but I don’t think he \emph{actually} had a mental model of himself continuing forever in the afterlife…”

The temperature of the room seemed to be dropping.

“You perceive,” said a voice like ice from the other end of the table, “that Dumbledore does not truly believe as he speaks. It is not that he has compromised his principles. It is that he never had them from the beginning. Are you becoming cynical yet, Mr~Potter?”

Harry had dropped his eyes to his teacup. “A little,” Harry said to his possibly-ultra-high-quality, perhaps-ridiculously-expensive Chinese tea. “I’m certainly becoming a bit \emph{frustrated} with…whatever’s going wrong in people’s heads.”

“Yes,” said that icy voice. “I find it frustrating as well.”

“Is there any way to get people \emph{not} to do that?” said Harry to his teacup.

“There is indeed a certain useful spell which solves the problem.”

Harry looked up hopefully at that, and saw a cold, cold smile on the Defence Professor’s face.

Then Harry got it. “I mean, \emph{besides} Avada Kedavra.”

The Defence Professor laughed. Harry didn’t.

“Anyway,” Harry said hastily, “I \emph{did} think fast enough not to suggest the obvious idea about the Resurrection Stone in front of Dumbledore. Have you ever seen a stone with a line, inside a circle, inside a triangle?”

The deathly chill seemed to draw back, fold into itself, as the ordinary Professor Quirrell returned. “Not that I can recall,” Professor Quirrell said after a while, a thoughtful frown on his face. “That is the Resurrection Stone?”

Harry set aside his teacup, then drew on his saucer the symbol he had seen on the inside of his cloak. And before Harry could take out his own wand to cast the Hover Charm, the saucer went floating obligingly across the table toward Professor Quirrell. Harry really wanted to learn that wandless stuff, but that, apparently, was far above his current curriculum.

Professor Quirrell studied Harry’s tea-saucer for a moment, then shook his head; and a moment later, the saucer went floating back to Harry.

Harry put his teacup back on the saucer, noting absently as he did so that the symbol he’d drawn had vanished. “If you happen to see a stone with that symbol,” said Harry, “and it \emph{does} talk to the afterlife, do let me know. I have a few questions for Merlin or anyone who was around in Atlantis.”

“Quite,” said Professor Quirrell. Then the Defence Professor lifted up his teacup again, and tipped it back as though to finish the last of what was there. “By the way, Mr~Potter, I fear we shall have to cut short today’s visit to Diagon Alley. I was hoping it would—but never mind. Let it stand that there is something else I must do this afternoon.”

Harry nodded, and finished his own tea, then rose from his seat at the same time as Professor Quirrell.

“One last question,” Harry said, as Professor Quirrell’s coat lifted itself off the coat rack and went floating toward the Defence Professor. “Magic is loose in the world, and I no longer trust my guesses so much as I once did. So in your own best guess and without any wishful thinking, do \emph{you} believe there’s an afterlife?”

“If I did, Mr~Potter,” said Professor Quirrell as he shrugged on his coat, “would I still be \emph{here}?”

%  LocalWords:  arry
