\chapter{Positive Bias}

\begin{chapterOpeningAuthorNote}
All these worlds are J.~K.~Rowling's, except Europa. Attempt no fanfics there.

One alert reviewer asked whether, if Luna is a seer, that means this is going to be an HPDM bottom!Draco mpreg fic. I regret that FFN does not allow me any larger font size in which to say NO. It honestly hadn't occurred to me that Luna might be a \emph{real} seer—I'll have to decide whether to run with that or not—but I think we can all safely assume that if Luna \emph{is} a seer, she said something about “light planting a seed in darkness”, and Xenophilius, as always, interpreted this in rather the wrong way.
\end{chapterOpeningAuthorNote}
\begin{chapterOpeningQuote}
Allow me to warn you that challenging my ingenuity is a dangerous sort of project, and may tend to make your life a lot more surreal.
\end{chapterOpeningQuote}

\lettrine{N}{o}-one had asked for help, that was the problem. They’d just gone around talking, eating, or staring into the air while their parents exchanged gossip. For whatever odd reason, no-one had been sitting down reading a book, which meant she couldn’t just sit down next to them and take out her own book. And even when she’d boldly taken the initiative by sitting down and continuing her third read-through of \emph{Hogwarts: A History,} no-one had seemed inclined to sit down next to her.

Aside from helping people with their homework, or anything else they needed, she really didn’t know how to meet people. She didn’t \emph{feel} like she was a shy person. She thought of herself as a take-charge sort of girl. And yet, somehow, if there wasn’t some request along the lines of “I can’t remember how to do long division” then it was just too \emph{awkward} to go up to someone and say…what? She’d never been able to figure out what. And there didn’t seem to be a standard information sheet, which was ridiculous. The whole business of meeting people had never seemed sensible to her. Why did \emph{she} have to take all the responsibility herself when there were two people involved? Why didn’t adults ever help? She wished some other girl would just walk up to \emph{her} and say, “Hermione, the teacher told me to be friends with you.”

But let it be quite clear that Hermione Granger, sitting alone on the first day of school in one of the few compartments that had been empty, in the last carriage of the train, with the compartment door left open just in case anyone for any reason wanted to talk to her, was \emph{not} sad, lonely, gloomy, depressed, despairing, or obsessing about her problems. She was, rather, rereading \emph{Hogwarts: A History} for the third time and quite enjoying it, with only a faint tinge of annoyance in the back of her mind at the general unreasonableness of the world.

There was the sound of an inter-train door opening, and then footsteps and an odd slithering sound coming down the hallway of the train. Hermione laid aside \emph{Hogwarts: A History} and stood up and stuck her head outside—just in case someone needed help—and saw a young boy in a wizard’s dress robes, probably first or second year going by his height, and looking quite silly with a scarf wrapped around his head. A small trunk stood on the floor next to him. Even as she saw him, he knocked on the door of another, closed compartment, and he said in a voice only slightly muffled by the scarf, “Excuse me, can I ask a quick question?”

She didn’t hear the answer from inside the compartment, but after the boy opened the door, she did think she heard him say—unless she’d somehow misheard— “Does anyone here know the six quarks or where I can find a first-year girl named Hermione Granger?”

After the boy had closed that compartment door, Hermione said, “Can I help you with something?”

The scarfed face turned to look at her, and the voice said, “Not unless you can name the six quarks or tell me where to find Hermione Granger.”

“Up, down, strange, charm, truth, beauty, and why are you looking for her?”

It was hard to tell from this distance, but she thought she saw the boy grin widely under his scarf. “Ah, so \emph{you’re} a first-year girl named Hermione Granger,” said that young, muffled voice. “On the train to Hogwarts, no less.” The boy started to walk towards her and her compartment, and his trunk slithered along after him. “Technically, all I needed to do was \emph{look} for you, but it seems likely that I’m meant to talk to you or invite you to join my party or get a key magical item from you or find out that Hogwarts was built over the ruins of an ancient temple or something. PC or NPC, that is the question?”

Hermione opened her mouth to reply to this, but then she couldn’t think of any \emph{possible} reply to…\emph{whatever} it was she’d just heard, even as the boy walked over to her, looked inside the compartment, nodded with satisfaction, and sat down on the bench across from her own. His trunk scurried in after him, grew to three times its former diameter and snuggled up next to her own in an oddly disturbing fashion.

“Please, have a seat,” said the boy, “and do please close the door behind you, if you would. Don’t worry, I don’t bite anyone who doesn’t bite me first.” He was already unwinding the scarf from around his head.

The imputation that this boy thought she was \emph{scared} of him made her hand send the door sliding shut, jamming it into the wall with unnecessary force. She spun around and saw a young face with bright, laughing green eyes, and an angry red-dark scar set into his forehead that reminded her of something in the back of her mind but right now she had more important things to think about. “I didn’t say I was Hermione Granger!”

“\emph{I} didn’t say you \emph{said} you were Hermione Granger, I just said you were Hermione Granger. If you’re asking how I know, it’s because I know everything. Good evening ladies and gentlemen, my name is Harry James Potter-Evans-Verres or Harry Potter for short, I know that probably doesn’t mean anything to \emph{you} for a change—”

Hermione’s mind finally made the connection. The scar on his forehead, the shape of a lightning bolt. “Harry Potter! You’re in \emph{Modern Magical History} and \emph{The Rise and Fall of the Dark Arts} and \emph{Great Wizarding Events of the Twentieth Century.}” It was actually the very first time in her whole life that she’d \emph{met} someone from inside a \emph{book}, and it was a rather odd feeling.

The boy blinked three times. “I’m in \emph{books}? Wait, of course I’m in books…what a strange thought.”

“Goodness, didn’t you know?” said Hermione. “I’d have found out everything I could if it was me.”

The boy spoke rather dryly. “Miss~Granger, it has been less than 72~hours since I went to Diagon Alley and discovered my claim to fame. I have spent the last two days buying science books. \emph{Believe me,} I intend to find out everything I can.” The boy hesitated. “What \emph{do} the books say about me?”

Hermione Granger’s mind flashed back, she hadn’t realised she would be tested on \emph{those} books so she’d read them only once, but it was just a month ago so the material was still fresh in her mind. “You’re the only one who’s survived the Killing Curse so you’re called the Boy-Who-Lived. You were born to James Potter and Lily Potter formerly Lily Evans on the 31st of July 1980. On the 31st of October 1981 the Dark Lord He-Who-Must-Not-Be-Named though I don’t know why not attacked your home. You were found alive with the scar on your forehead in the ruins of your parents’ house near the burnt remains of You-Know-Who’s body. Chief Warlock Albus Percival Wulfric Brian Dumbledore sent you off somewhere, no-one knows where. \emph{The Rise and Fall of the Dark Arts} claims that you survived because of your mother’s love and that your scar contains all of the Dark Lord’s magical power and that the centaurs fear you, but \emph{Great Wizarding Events of the Twentieth Century} doesn’t mention anything like that and \emph{Modern Magical History} warns that there are lots of crackpot theories about you.”

The boy’s mouth was hanging open. “Were you told to wait for Harry Potter on the train to Hogwarts, or something like that?”

“No,” Hermione said. “Who told you about \emph{me?}”

“Professor McGonagall and I believe I see why. Do you have an eidetic memory, Hermione?”

Hermione shook her head. “It’s not photographic, I’ve always wished it was but I had to read my school books five times over to memorize them all.”

“Really,” the boy said in a slightly strangled voice. “I hope you don’t mind if I test that—it’s not that I don’t believe you, but as the saying goes, ‘Trust, but verify’. No point in wondering when I can just do the experiment.”

Hermione smiled, rather smugly. She so loved tests. “Go ahead.”

The boy stuck a hand into a pouch at his side and said “Magical Drafts and Potions by Arsenius Jigger”. When he withdrew his hand it was holding the book he’d named.

Instantly Hermione wanted one of those pouches more than she’d ever wanted anything.

The boy opened the book to somewhere in the middle and looked down. “If you were making \emph{oil of sharpness}—”

“I can \emph{see} that page from here, you know!”

The boy tilted the book so that she couldn’t see it any more, and flipped the pages again. “If you were brewing a \emph{potion of spider climbing,} what would be the next ingredient you added after the Acromantula silk?”

“After dropping in the silk, wait until the potion has turned exactly the shade of the cloudless dawn sky, 8 degrees from the horizon and 8 minutes before the tip of the sun first becomes visible. Stir eight times widdershins and once deasil, and then add eight drams of unicorn bogies.”

The boy shut the book with a sharp snap and put the book back into his pouch, which swallowed it with a small burping noise. “Well well well \emph{well} well well. I should like to make you a proposition, Miss~Granger.”

“A proposition?” Hermione said suspiciously. Girls weren’t supposed to listen to those.

It was also at this point that Hermione realised the other thing—well, one of the things—which was odd about the boy. Apparently people who were \emph{in} books actually \emph{sounded} like a book when they talked. This was quite the surprising discovery.

The boy reached into his pouch and said, “can of pop”, retrieving a bright green cylinder. He held it out to her and said, “Can I offer you something to drink?”

Hermione politely accepted the fizzy drink. In fact she \emph{was} feeling sort of thirsty by now. “Thank you very much,” Hermione said as she popped the top. “Was that your proposition?”

The boy coughed. “No,” he said. Just as Hermione started to drink, he said, “I’d like you to help me take over the universe.”

Hermione finished her drink and lowered the can. “No thank you, I’m not evil.”

The boy looked at her in surprise, as though he’d been expecting some other answer. “Well, I was speaking a bit rhetorically,” he said. “In the sense of the Baconian project, you know, not political power. ‘The effecting of all things possible’ and so on. I want to conduct experimental studies of spells, figure out the underlying laws, bring magic into the domain of science, merge the wizarding and Muggle worlds, raise the entire planet’s standard of living, move humanity centuries ahead, discover the secret of immortality, colonize the Solar System, explore the galaxy, and most importantly, figure out what the heck is really going on here because all of this is blatantly impossible.”

That sounded a bit more interesting. “And?”

The boy stared at her incredulously. “\emph{And?} That’s not \emph{enough?}”

“And what do you want from me?” said Hermione.

“I want you to help me do the research, of course. With your encyclopedic memory added to my intelligence and rationality, we’ll have the Baconian project finished in no time, where by ‘no time’ I mean probably at least thirty-five years.”

Hermione was beginning to find this boy annoying. “I haven’t seen you do anything intelligent. Maybe I’ll let \emph{you} help me with \emph{my} research.”

There was a certain silence in the compartment.

“So you’re asking me to demonstrate my intelligence, then,” said the boy after a long pause.

Hermione nodded.

“I warn you that challenging my ingenuity is a dangerous project, and tends to make your life a lot more surreal.”

“I’m not impressed yet,” Hermione said. Unnoticed, the green drink once again rose to her lips.

“Well, maybe \emph{this} will impress you,” the boy said. He leaned forward and looked at her intensely. “I’ve already done a bit of experimenting and I found out that I don’t need the wand, I can make anything I want happen just by snapping my fingers.”

It came just as Hermione was in the middle of swallowing, and she choked and coughed and expelled the bright green fluid.

Onto her brand new, never-before-worn witch’s robes, on the very first day of school.

Hermione actually screamed. It was a high-pitched sound that sounded like an air raid siren in the closed compartment. “\emph{Eek! My clothes!}”

“Don’t panic!” said the boy. “I can fix it for you. Just watch!” He raised a hand and snapped his fingers.

“You’ll—” Then she looked down at herself.

The green fluid was still there, but even as she watched, it started to vanish and fade and within just a few moments, it was like she’d never spilled anything at herself.

Hermione stared at the boy, who was wearing a rather smug sort of smile.

Wordless wandless magic! At \emph{his} age? When he’d only got his schoolbooks \emph{three days} ago?

Then she remembered what she’d read, and she gasped and flinched back from him. \emph{All the Dark Lord’s magical power! In his scar!}

She rose hastily to her feet. “I, I, I need to go the toilet, wait here all right—” she had to find a grown-up she had to tell them—

The boy’s smile faded. “It was just a trick, Hermione. I’m sorry, I didn’t mean to scare you.”

Her hand halted on the door handle. “A \emph{trick?}”

“Yes,” said the boy. “You asked me to demonstrate my intelligence. So I did something apparently impossible, which is always a good way to show off. I can’t \emph{really} do anything just by snapping my fingers.” The boy paused. “At least I don’t \emph{think} I can, I’ve never actually tested it experimentally.” The boy raised his hand and snapped his fingers again. “Nope, no banana.”

Hermione was as confused as she’d ever been in her life.

The boy was now smiling again at the look on her face. “I did \emph{warn} you that challenging my ingenuity tends to make your life surreal. Do remember this the next time I warn you about something.”

“But, but,” Hermione stammered. “What did you \emph{do}, then?”

The boy’s gaze took on a measuring, weighing quality that she’d never seen before from someone her own age. “You think you have what it takes to be a scientist in your own right, with or without my help? Then let’s see how \emph{you} investigate a confusing phenomenon.”

“I…” Hermione’s mind went blank for a moment. She loved tests but she’d never had a test like \emph{this} before. Frantically, she tried to cast back for anything she’d read about what scientists were supposed to do. Her mind skipped gears, ground against itself, and spat back the instructions for doing a science investigation project:

\emph{Step 1: Form a hypothesis.\\
Step 2: Do an experiment to test your hypothesis.\\
Step 3: Measure the results.\\
Step 4: Make a cardboard poster.}

Step 1 was to form a hypothesis. That meant, try to think of something that \emph{could} have happened just now. “All right. My hypothesis is that you cast a Charm on my robes to make anything spilled on it vanish.”

“All right,” said the boy, “is that your answer?”

The shock was wearing off, and Hermione’s mind was starting to work properly. “Wait, that can’t be right. I didn’t see you touch your wand or say any spells so how could you have cast a Charm?”

The boy waited, his face neutral.

“But suppose all the robes come from the store with a Charm \emph{already} on them to keep them clean, which would be a useful sort of Charm for them to have. You found that out by spilling something on \emph{yourself} earlier.”

Now the boy’s eyebrows lifted. “Is \emph{that} your answer?”

“No, I haven’t done Step 2, ‘Do an experiment to test your hypothesis.’”

The boy closed his mouth again, and began to smile.

Hermione looked at the drinks can, which she’d automatically put into the cup-holder at the window. She took it up and peered inside, and found that it was around one-third full.

“Well,” said Hermione, “the experiment I want to do is to pour it on my robes and see what happens, and my prediction is that the stain will disappear. Only if it \emph{doesn’t} work, my robes will be stained, and I don’t want that.”

“Do it to mine,” said the boy, “that way you don’t have to worry about your robes getting stained.”

“But—” Hermione said. There was something \emph{wrong} with that thinking but she didn’t know how to say it exactly.

“I have spare robes in my trunk,” said the boy.

“But there’s nowhere for you to change,” Hermione objected. Then she thought better of it. “Though I suppose I could leave and close the door—”

“I have somewhere to change in my trunk, too.”

Hermione looked at his trunk, which, she was beginning to suspect, was rather more special than her own.

“All right,” Hermione said, “since you say so,” and she rather gingerly poured a bit of green pop onto a corner of the boy’s robes. Then she stared at it, trying to remember how long the original fluid had taken to disappear…

And the green stain vanished!

Hermione let out a sigh of relief, not least because this meant she wasn’t dealing with all of the Dark Lord’s magical power.

Well, Step 3 was measuring the results, but in this case that was just seeing that the stain had vanished. And she supposed she could probably skip Step 4, about the cardboard poster. “My answer is that the robes are Charmed to keep themselves clean.”

“Not quite,” said the boy.

Hermione felt a stab of disappointment. She really wished she \emph{wouldn’t} have felt that way, the boy wasn’t a teacher, but it was still a test and she’d got a question wrong and that always felt like a little punch in the stomach.

(It said almost everything you needed to know about Hermione Granger that she had never let that stop her, or even let it interfere with her love of being tested.)

“The sad thing is,” said the boy, “you probably did everything the book told you to. You made a prediction that would distinguish between the robe being charmed and not charmed, and you tested it, and rejected the null hypothesis that the robe was not charmed. But unless you read the very, very best sort of books, they won’t quite teach you how to do science \emph{properly}. Well enough to \emph{really} get the right answer, I mean, and not just churn out another publication like Dad always complains about. So let me try to explain—without giving away the answer—what you did wrong this time, and I’ll give you another chance.”

She was starting to resent the boy’s oh-so-superior tone when he was just another eleven-year-old like her, but that was secondary to finding out what she’d done wrong. “All right.”

The boy’s expression grew more intense. “This is a game based on a famous experiment called the 2–4–6 task, and this is how it works. I have a \emph{rule}—known to me, but not to you—which fits some triplets of three numbers, but not others. 2–4–6 is one example of a triplet which fits the rule. In fact…let me write down the rule, just so you know it’s a fixed rule, and fold it up and give it to you. Please don’t look, since I infer from earlier that you can read upside-down.”

The boy said “paper” and “mechanical pencil” to his pouch, and she shut her eyes tightly while he wrote.

“There,” said the boy, and he was holding a tightly folded piece of paper. “Put this in your pocket,” and she did.

“Now the way this game works,” said the boy, “is that you give me a triplet of three numbers, and I’ll tell you ‘Yes’ if the three numbers are an instance of the rule, and ‘No’ if they’re not. I am Nature, the rule is one of my laws, and you are investigating me. You already know that 2–4–6 gets a ‘Yes’. When you’ve performed all the further experimental tests you want—asked me as many triplets as you feel necessary—you stop and guess the rule, and then you can unfold the sheet of paper and see how you did. Do you understand the game?”

“Of course I do,” said Hermione.

“Go.”

“4–6–8” said Hermione.

“Yes,” said the boy.

“10–12–14”, said Hermione.

“Yes,” said the boy.

Hermione tried to cast her mind a little further afield, since it seemed like she’d already done all the testing she needed, and yet it couldn’t be that easy, could it?

“1–3–5.”

“Yes.”

“Minus 3, minus 1, plus 1.”

“Yes.”

Hermione couldn’t think of anything else to do. “The rule is that the numbers have to increase by two each time.”

“Now suppose I tell you,” said the boy, “that this test is harder than it looks, and that only 20\% of grown-ups get it right.”

Hermione frowned. What had she missed? Then, suddenly, she thought of a test she still needed to do.

“2–5–8!” she said triumphantly.

“Yes.”

“10–20–30!”

“Yes.”

“The real answer is that the numbers have to go up by the \emph{same} amount each time. It doesn’t have to be 2.”

“Very well,” said the boy, “take the paper out and see how you did.”

Hermione took the paper out of her pocket and unfolded it.

\emph{Three real numbers in increasing order, lowest to highest.}

Hermione’s jaw dropped. She had the distinct feeling of something terribly unfair having been done to her, that the boy was a dirty rotten cheating liar, but when she cast her mind back she couldn’t think of any wrong responses that he’d given.

“What you’ve just discovered is called ‘positive bias’,” said the boy. “You had a rule in your mind, and you kept on thinking of triplets that should make the rule say ‘Yes’. But you didn’t try to test any triplets that should make the rule say ‘No’. In fact you didn’t get a \emph{single} ‘No’, so ‘any three numbers’ could have just as easily been the rule. It’s sort of like how people imagine experiments that could confirm their hypotheses instead of trying to imagine experiments that could falsify them—that’s not quite exactly the same mistake but it’s close. You have to learn to look on the negative side of things, stare into the darkness. When this experiment is performed, only 20\% of grown-ups get the answer right. And many of the others invent fantastically complicated hypotheses and put great confidence in their wrong answers since they’ve done so many experiments and everything came out like they expected.”

“Now,” said the boy, “do you want to take another shot at the original problem?”

His eyes were quite intent now, as though this were the \emph{real} test.

Hermione shut her eyes and tried to concentrate. She was sweating underneath her robes. She had an odd feeling that this was the hardest she’d ever been asked to think on a test or maybe even the \emph{first} time she’d ever been asked to think on a test.

What other experiment could she do? She had a Chocolate Frog, could she try to rub some of that on the robes and see if \emph{it} vanished? But that still didn’t seem like the kind of twisty negative thinking the boy was asking for. Like she was still asking for a ‘Yes’ if the Chocolate Frog stain disappeared, rather than asking for a ‘No’.

So…on her hypothesis…when should the pop…\emph{not} vanish?

“I have an experiment to do,” Hermione said. “I want to pour some pop on the floor, and see if it \emph{doesn’t} vanish. Do you have some paper towels in your pouch, so I can mop up the spill if this doesn’t work?”

“I have napkins,” said the boy. His face still looked neutral.

Hermione took the can, and poured a small bit of pop onto the floor.

A few seconds later, it vanished.

Then the realisation hit her and she felt like kicking herself. “Of course! \emph{You} gave me that can! It’s not the robe that’s enchanted, it was the pop all along!”

The boy stood up and bowed to her solemnly. He was grinning widely now. “Then…may I help you with your research, Hermione Granger?”

“I, ah…” Hermione was still feeling the rush of euphoria, but she wasn’t quite sure about how to answer \emph{that}.

They were interrupted by a weak, tentative, faint, rather \emph{reluctant} knocking at the door.

The boy turned and looked out the window, and said, “I’m not wearing my scarf, so can you get that?”

It was at this point that Hermione realised why the boy—no, the Boy-Who-Lived, Harry Potter—had been wearing the scarf over his head in the first place, and felt a little silly for not realising it earlier. It was actually sort of odd, since she would have thought Harry Potter would proudly display himself to the world; and the thought occurred to her that he might actually be shyer than he seemed.

When Hermione pulled the door open, she was greeted by a trembling young boy who looked exactly like he knocked.

“Excuse me,” said the boy in a tiny voice, “I’m Neville Longbottom. I’m looking for my pet toad, I, I can’t seem to find it anywhere on this carriage…have you seen my toad?”

“No,” Hermione said, and then her helpfulness kicked in full throttle. “Have you checked all the other compartments?”

“Yes,” whispered the boy.

“Then we’ll just have to check all the other carriages,” Hermione said briskly. “I’ll help you. My name is Hermione Granger, by the way.”

The boy looked like he might faint with gratitude.

“Hold on,” came the voice of the \emph{other} boy—Harry Potter. “I’m not sure that’s the best way to do it.”

At this Neville looked like he might cry, and Hermione swung around, angered. If Harry Potter was the sort of person who’d abandon a little boy just because he didn’t want to be interrupted… “What? Why \emph{not?}”

“Well,” said Harry Potter, “It’s going to take a while to check the whole train by hand, and we might miss the toad anyway, and if we didn’t find it by the time we’re at Hogwarts, he’d be in trouble. So what would make a lot more sense is if he went directly to the front carriage, where the prefects are, and asked a prefect for help. That was the first thing I did when I was looking for you, Hermione, although they didn’t actually know. But they might have spells or magic items that would make it a lot easier to find a toad. We’re only first-years.”

That…\emph{did} make a lot of sense.

“Do you think you can make it to the prefects’ carriage on your own?” asked Harry Potter. “I’ve sort of got reasons for not wanting to show my face too much.”

Suddenly Neville gasped and took a step back. “I remember that voice! You’re one of the Lords of Chaos! \emph{You’re the one who gave me chocolate!}”

What? What what \emph{what?}

Harry Potter turned his head from the window and rose dramatically. “I \emph{never!}” he said, voice full of indignation. “Do I look like the sort of villain who would give sweets to a child?”

Neville’s eyes widened. “\emph{You’re} Harry Potter? \emph{The} Harry Potter? \emph{You?}”

“No, just \emph{a} Harry Potter, there are three of me on this train—”

Neville gave a small shriek and ran away. There was a brief pattering of frantic footsteps and then the sound of a carriage door opening and closing.

Hermione sat down hard on her bench. Harry Potter closed the door and then sat down next to her.

“Can you please explain to me what’s going on?” Hermione said in a weak voice. She wondered if hanging around Harry Potter meant always being this confused.

“Oh, well, what happened was that Fred and George and I saw this poor small boy at the train station—the woman next to him had gone away for a bit, and he was looking really frightened, like he was sure he was about to be attacked by Death Eaters or something. Now, there’s a saying that the fear is often worse than the thing itself, so it occurred to me that this was a lad who could actually benefit from seeing his worst nightmare come true and that it wasn’t so bad as he feared—”

Hermione sat there with her mouth wide open.

“—and Fred and George came up with this spell to make the scarves over our faces darken and blur, like we were undead kings and those were our grave shrouds—”

She didn’t like at all where this was going.

“—and after we were done giving him all the sweets I’d bought, we were like, ‘Let’s give him some money! Ha ha ha! Have some Knuts, boy! Have a silver Sickle!’ and dancing around him and laughing evilly and so on. I think there were some people in the crowd who wanted to interfere at first, but bystander apathy held them off at least until they saw what we were doing, and then I think they were all too confused to do anything. Finally he said in this tiny little whisper ‘go away’ so the three of us all screamed and ran off, shrieking something about the light burning us. Hopefully he won’t be as scared of being bullied in the future. That’s called desensitisation therapy, by the way.”

Okay, she \emph{hadn’t} guessed right about where this was going.

The burning fire of indignation that was one of Hermione’s primary engines sputtered into life, even though part of her \emph{did} sort of see what they’d been trying to do. “That’s awful! \emph{You’re} awful! That poor boy! What you did was \emph{mean!}”

“I think the word you’re looking for is \emph{enjoyable}, and in any case you’re asking the wrong question. The question is, did it do more good than harm, or more harm than good? If you have any arguments to contribute to \emph{that} question I’m glad to hear them, but I won’t entertain any other criticisms until that one is settled. I certainly agree that what I did \emph{looks} all terrible and bullying and mean, since it involves a scared little boy and so on, but that’s hardly the key issue now is it? That’s called \emph{consequentialism}, by the way, it means that whether an act is right or wrong isn’t determined by whether it \emph{looks} bad, or mean, or anything like that, the only question is how it will turn out in the end—what are the consequences.”

Hermione opened her mouth to say something utterly \emph{searing} but unfortunately she seemed to have neglected the part where she thought of something to say before opening her mouth. All she could come up with was, “What if he has \emph{nightmares?}”

“Honestly, I don’t think he needed our help to have nightmares, and if he has nightmares about \emph{this} instead, then it’ll be nightmares involving horrible monsters who give you chocolate and that was sort of the whole \emph{point}.”

Hermione’s brain kept hiccuping in confusion every time she tried to get properly angry. “Is your life always this peculiar?” she said at last.

Harry Potter’s face gleamed with pride. “I \emph{make} it that peculiar. You’re looking at the product of a lot of hard work and elbow grease.”

“So…” Hermione said, and trailed off awkwardly.

“So,” Harry Potter said, “how much science do you know exactly? I can do calculus and I know some Bayesian probability theory and decision theory and a lot of cognitive science, and I’ve read \emph{The Feynman Lectures} (or volume 1 anyway) and \emph{Judgment Under Uncertainty: Heuristics and Biases} and \emph{Language in Thought and Action} and \emph{Influence: Science and Practice} and \emph{Rational Choice in an Uncertain World} and \emph{Gödel, Escher, Bach} and \emph{A Step Farther Out} and—”

The ensuing quiz and counter-quiz went on for several minutes before being interrupted by another timid knock at the door. “Come in,” she and Harry Potter said at almost the same time, and it slid back to reveal Neville Longbottom.

Neville \emph{was} actually crying now. “I went to the front carriage and found a p-prefect but he t-told me that prefects weren’t to be bothered over little things like m-missing toads.”

The Boy-Who-Lived’s face changed. His lips set in a thin line. His voice, when he spoke, was cold and grim. “What were his colours? Green and silver?”

“N-no, his badge was r-red and gold.”

“\emph{Red and gold!}” burst out Hermione. “But those are \emph{Gryffindor’s} colours!”

Harry Potter \emph{hissed} at that, a frightening sort of sound that could have come from a live snake and made both her and Neville flinch. “I \emph{suppose,}” Harry Potter spat, “that finding some first-year’s toad isn’t \emph{heroic} enough to be worthy of a \emph{Gryffindor} prefect. Come on, Neville, \emph{I’ll} come with you this time, we’ll see if the Boy-Who-Lived gets more attention. First we’ll find a prefect who ought to know a spell, and if that doesn’t work, we’ll find a prefect who isn’t afraid of getting their hands dirty, and if \emph{that} doesn’t work, I’ll start recruiting my fans and if we have to we’ll take apart the whole train screw by screw.”

The Boy-Who-Lived stood up and grabbed Neville’s hand in his, and Hermione realised with a sudden brain hiccough that they were nearly the same size, even though some part of her had insisted that Harry Potter was a foot taller than that, and Neville at least six inches shorter.

“\emph{Stay!}” Harry Potter snapped at her—no, wait, at his \emph{trunk}—and he closed the door behind him firmly as he left.

She probably should have gone with them, but in just a brief moment Harry Potter had turned so scary that she was actually rather glad she hadn’t thought to suggest it.

% The misspelling ``History: A Hogwarts'' is intentional.
Hermione’s mind was now so jumbled that she didn’t even think she could properly read \emph{History: A Hogwarts}. She felt as if she’d just been run over by a steamroller and turned into a pancake. She wasn’t sure what she was thinking or what she was feeling or why. She just sat by the window and stared at the moving scenery.

Well, she did at least know why she was feeling a little sad inside.

Maybe Gryffindor wasn’t as wonderful as she had thought.

%  LocalWords:  NPC Eek Judgment
